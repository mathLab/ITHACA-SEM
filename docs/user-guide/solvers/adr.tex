\section{ADRSolver}

%3.4/UserGuide/Tutorial/ADRSolver
%3.4/UserGuide/Examples/ADRSolver/1DAdvection
%3.4/UserGuide/Examples/ADRSolver/3DAdvectionMassTransport
%3.4/UserGuide/Examples/ADRSolver/Helmholtz2D

\subsection{Synopsis}

The ADRSolver is designed to solve partial differential equations of the form:
\begin{equation}
\alpha \dfrac{\partial u}{\partial t} + \lambda u + \nu \nabla u + \epsilon \nabla \cdot (D \nabla u) = f
\end{equation}
in either discontinuous or continuous projections of the solution field. 
For a full list of the equations which are supported, and the capabilities of each equation, 
see the table below.

\begin{table}[h!]
\begin{center}
\tiny
\renewcommand\arraystretch{2.2} 
\begin{tabular}{|l|l|l|l|}
\hline
\textbf{Equation to solve}			    								         & \textbf{EquationType} 	                                 & \textbf{Dimensions supported}   & \textbf{Projection supported} \\
\hline 
$\nabla^2 u = 0$      													& Laplace 	  		                                 & All 	   	  			&  Continuous/Discontinuous	\\
\hline
$\nabla^2 u  =  f$       					                                    		        & Poisson 	  	                                                  & All 	   	  			&  Continuous/Discontinuous	\\
\hline
$\nabla^2 u  + \lambda u =  f$     					                                         & Helmholtz 	  	                                          & All 	   	 			 &  Continuous/Discontinuous	\\
\hline
$\epsilon \nabla^2 u + \mathbf{V}\nabla u = f$    						        & SteadyAdvectionDiffusion 	  	                 & 2D only 	   	 		 &  Continuous/Discontinuous	\\
\hline
$\epsilon \nabla^2 u +  \lambda u = f$       								        & SteadyDiffusionReaction 	  	                 & 2D only 	   	 		 &  Continuous/Discontinuous	\\
\hline
$\epsilon \nabla^2 u  \mathbf{V}\nabla u + \lambda u = f$   			         	& SteadyAdvectionDiffusionReaction 	  	& 2D only 	   	 		 &  Continuous/Discontinuous	\\
\hline
$ \dfrac{\partial u}{\partial t} + \mathbf{V}\nabla u = f$       		                                  & UnsteadyAdvection 	  	                          & All 	   			 &  Continuous/Discontinuous	\\
\hline
$\dfrac{\partial u}{\partial t}  = \epsilon \nabla^2 u$      			                         & UnsteadyDiffusion 	  	                                 & All 	   	  			&  Continuous/Discontinuous	\\
\hline
$\dfrac{\partial u}{\partial t}  + \mathbf{V}\nabla u = \epsilon \nabla^2 u$        		& UnsteadyAdvectionDiffusion 	  	                & All 	   				  &  Continuous/Discontinuous	\\
\hline
$\dfrac{\partial u}{\partial t}  + u\nabla u =  0$       		                                          & UnsteadyInviscidBurger 	  	                & 1D only 	   			  &  Continuous/Discontinuous	\\
\hline
\end{tabular}
\end{center}
\caption{Equations supported by the ADRSolver with their capabilities.}
\label{t:ADR1}
\end{table}


\subsection{Usage}

ADRSolver session.xml

\subsection{Session file configuration}

The type of equation which is to be solved is specified through the EquationType 
SOLVERINFO option in the session file. This can be set as in table \ref{t:ADR1}.
At present, the Steady non-symmetric solvers cannot be used in parallel. \\

\subsubsection{Solver Info}

The solver info are listed below:
\begin{itemize}
\item \textbf{Eqtype}: This sets the type of equation to solve, according to the table above.
\item \textbf{TimeIntegrationMethod}: The following types of time integration methods have been tested with each solver:
\begin{table}[h!]
\begin{center}
\footnotesize
\renewcommand\arraystretch{1.2} 
\begin{tabular}{|c|c|c|c|c|}
\hline
& \textbf{Explicit} & \textbf{Diagonally Implicit}   & \textbf{ IMEX}  &  \textbf{Implicit}   \\
\hline 
UnstedayAdvection     				&  \checkmark 	 &	 & 	&\\
\hline
UnstedayDifusion   					&  \checkmark 	 & \checkmark 	 & 	&\\
\hline
UnstedayAdvectionDiffusion     			&  	 &   	& \checkmark	&\\
\hline
UnstedayInviscidBurger     			&  \checkmark 	 &	 & 	&\\
\hline

\end{tabular}
\end{center}
\label{t:ADR2}
\end{table}
\vspace{-1 cm}
\item \textbf{Projection}: The Galerkin projection used may be either
\begin{itemize}
	\item Continuous for a C0-continuous Galerkin (CG) projection.
	\item Discontinuous for a discontinous Galerkin (DG) projection.
\end{itemize}
\item \textbf{DiffusionAdvancement}: This specifies how to treat the diffusion term. This will be restricted by the choice of time integration scheme.
\begin{itemize}
	\item Explicit: Requires the use of an explicit time integration scheme.
	\item Implcit: Requires the use of a diagonally implicit, IMEX or Implicit scheme.
\end{itemize}
\item \textbf{AdvectionAdvancement}: This specifies how to treat the advection term. This will be restricted by the choice of time integration scheme
\begin{itemize}
	\item Explicit: Requires the use of an explicit or IMEX time integration scheme.
	\item Implicit: Not supported at present.
\end{itemize}
\item \textbf{AdvectionType}: Specifies the type of advection
\begin{itemize}
	\item NonConservative (for CG only).
	\item WeakDG (for DG only).
\end{itemize}
\item \textbf{DiffusionType}:
\begin{itemize}
	\item LDG.
\end{itemize}
\item \textbf{UpwindType}:
\begin{itemize}
	\item Upwind.
\end{itemize}
\end{itemize}

\subsubsection{Parameters}

The following parameters can be specified in the PARAMETERS section of the session file:
\begin{itemize}
\item \textbf{epsilon}: sets the diffusion coefficient $\epsilon$.\\ 
\textit{Can be used} in: SteadyDiffusionReaction, SteadyAdvectionDiffusionReaction, UnsteadyDiffusion, UnsteadyAdvectionDiffusion. \\
\textit{Default value}: 0.
\item  \textbf{d00, d11, d22}: sets the diagonal entries of the diffusion tensor formula $D$. \\
\textit{Can be used in}: UnsteadyDiffusion \\
\textit{Default value}: All set to 1 (i.e. identity matrix). 
\item  \textbf{lambda}: sets the reaction coefficient formula $\lambda$. \\
\textit{Can be used in}: SteadyDiffusionReaction, Helmholtz, SteadyAdvectionDiffusionReaction\\
\textit{Default value}: 0.
\end{itemize}

\subsubsection{Functions}

The following functions can be specified inside the CONDITIONS section of the session file:

\begin{itemize}
\item \textbf{AdvectionVelocity}: specifies the advection velocity formula.
\item \textbf{InitialConditions}: specifies the initial condition for unsteady problems.
\item \textbf{Forcing}: specifies the forcing function formula
\end{itemize}

\subsection{Examples}
\subsubsection{1D Advection equation}

In this example, it will be demonstrated how the Advection equation can be solved on a one-dimensional domain.
This problem is a particular case of the Advection-Diffusion-Reaction Solver. \\

\textbf{Advection equation}

We consider the hyperbolic partial differential equation:
\begin{equation}
\dfrac{\partial u}{\partial t} + \dfrac{\partial f}{\partial x} = 0,
\end{equation}
where $f =  a u$ i s the advection flux.

\textbf{Input file}

Advection1D.xml

\textbf{\footnotesize{Geometry definition}}

\begin{lstlisting}[style=XMLStyle]
<GEOMETRY DIM="1" SPACE="1">
        <VERTEX>
            <V ID="0"> -1.0  0.0  0.0</V>
            <V ID="1"> -0.8  0.0  0.0</V>
            <V ID="2"> -0.6  0.0  0.0</V>
            <V ID="3"> -0.4  0.0  0.0</V>
            <V ID="4"> -0.2  0.0  0.0</V>
            <V ID="5">  0.0  0.0  0.0</V>
            <V ID="6">  0.2  0.0  0.0</V>
            <V ID="7">  0.4  0.0  0.0</V>
            <V ID="8">  0.6  0.0  0.0</V>
            <V ID="9">  0.8  0.0  0.0</V>
            <V ID="10"> 1.0  0.0  0.0</V>
        </VERTEX> 
        
        <ELEMENT>
            <S ID="0">    0     1 </S>
            <S ID="1">    1     2 </S>
            <S ID="2">    2     3 </S>
            <S ID="3">    3     4 </S>
            <S ID="4">    4     5 </S>
            <S ID="5">    5     6 </S>
            <S ID="6">    6     7 </S>
            <S ID="7">    7     8 </S>
            <S ID="8">    8     9 </S>
            <S ID="9">    9    10 </S>
        </ELEMENT>
        
        <COMPOSITE>
            <C ID="0"> S[0-9] </C>
            <C ID="1"> V[0]   </C>
            <C ID="2"> V[10]  </C>
        </COMPOSITE>
        
        <DOMAIN> C[0] </DOMAIN>
</GEOMETRY>
\end{lstlisting}

\textbf{\footnotesize{Expansion definition}}

\begin{lstlisting}[style=XMLStyle]
<EXPANSIONS>
        <E COMPOSITE="C[0]" FIELDS="u" TYPE="GLL_LAGRANGE_SEM" NUMMODES="4"/>
</EXPANSIONS>
\end{lstlisting}

\textbf{\footnotesize{Conditions definition}}

\begin{lstlisting}[style=XMLStyle]
<CONDITIONS>
    
        <PARAMETERS>
            <P> FinTime         = 20                     </P>
            <P> TimeStep        = 0.01                 </P>
            <P> NumSteps        = FinTime/TimeStep      </P>
            <P> IO_CheckSteps   = 100000                </P>
            <P> IO_InfoSteps    = 100000                </P>
            <P> advx            = 1                     </P>
            <P> advy            = 0                     </P>
        </PARAMETERS>
        
        <SOLVERINFO>
            <I PROPERTY="EQTYPE"                VALUE="UnsteadyAdvection"   />
            <I PROPERTY="Projection"            VALUE="DisContinuous"       />
            <I PROPERTY="AdvectionType"         VALUE="FRDG"                />
            <I PROPERTY="UpwindType"            VALUE="Upwind"              />
            <I PROPERTY="TimeIntegrationMethod" VALUE="ClassicalRungeKutta4"/>
        </SOLVERINFO>

        <VARIABLES>
            <V ID="0"> u </V>
        </VARIABLES>

        <BOUNDARYREGIONS>
            <B ID="0"> C[1] </B>
            <B ID="1"> C[2] </B>
        </BOUNDARYREGIONS>

        <BOUNDARYCONDITIONS>
            <REGION REF="0">
                <P VAR="u" VALUE="[1]" />
            </REGION>
            <REGION REF="1">
                <P VAR="u" VALUE="[0]" />
            </REGION>
        </BOUNDARYCONDITIONS>

        <FUNCTION NAME="AdvectionVelocity">
            <E VAR="Vx" VALUE="advx" />
        </FUNCTION>
        
        <FUNCTION NAME="InitialConditions">
            <E VAR="u" VALUE="exp(-20.0*x*x)" />
        </FUNCTION>

        <FUNCTION NAME="ExactSolution">
            <E VAR="u" VALUE="exp(-20.0*x*x)" />
        </FUNCTION>

</CONDITIONS>
\end{lstlisting}

\textbf{Running the code}

ADRSolver Advection1D.xml

\textbf{Post-processing}

FldToTecplot Advection1D.xml Advection1D.fld \\

FldToVtk Advection1D.xml Advection1D.fld



\subsubsection{2D Helmholtz Problem}

In this example, it will be demonstrated how the Helmholtz equation can be solved on a two-dimensional domain. 
This problem is a particular case of the Advection-Diffusion-Reaction Solver.

\textbf{Helmholtz equation}

We consider the elliptic partial differential equation:

\begin{equation}
\nabla^2 u  + \lambda u =  f
\end{equation}

where $\nabla^2$ is the Laplacian and $\lambda$ is a real positive constant.

\textbf{Input file} 

For this tutorial, the input file (in the \nekpp input format) used can be found
 in  \nekpp/regressionTests/Solvers/ADRSolver/InputFiles/Test\_Helmholtz2D\_modal.xml.

\textbf{\footnotesize{Geometry definition}}
In the GEOMETRY section, the dimensions of the problem are defined. 
Then, the coordinates (XSCALE, YSCALE, ZSCALE) of each vertices are specified. As this input file defines a two-dimensional problem: ZSCALE = 0.

\begin{lstlisting}[style=XMLStyle]
<GEOMETRY DIM="2" SPACE="2">

    <VERTEX>
        <V ID="0"> -1.000000000000000 3.500000000000000 0.0 </V>
        .
        .
        <V ID="48"> 3.179196984040000 2.779077508740000 0.0 </V>
    </VERTEX>
\end{lstlisting}

Edges can now be defined by two vertices.

\begin{lstlisting}[style=XMLStyle]
 <EDGE>
        <E ID="0"> 1 21 </E>
        .
        .
        <E ID="95"> 10 19 </E>
    </EDGE>
\end{lstlisting}

In the ELEMENT section, the tag T and Q define respectively triangular and quadrilateral element.
 Triangular elements are defined by a sequence of three edges and quadrilateral elements by a sequence of four edges.

\begin{lstlisting}[style=XMLStyle]
<ELEMENT>
        <T ID="0"> 6 7 16 </T>
        .
        .
        <T ID="21"> 29 37 34 </T>
        <Q ID="22"> 35 41 46 40 </Q>
        .
        .
        <Q ID="47"> 91 92 95 93 </Q>
    </ELEMENT>
\end{lstlisting}

Finally, collections of elements are listed in the COMPOSITE section and the DOMAIN section specifies that 
the mesh is composed by all the triangular and quadrilateral elements. The other composites will be used to enforce boundary conditions.

\begin{lstlisting}[style=XMLStyle]
 <COMPOSITE>
        <C ID="0"> Q[22-47] </C>
        <C ID="1"> T[0-21] </C>
        <C ID="2"> E[0-1] </C>
        .
        .
        <C ID="10"> E[84,75,69,62,51,40,30,20,6] </C>
    </COMPOSITE>

    <DOMAIN> C[0-1] </DOMAIN>
</GEOMETRY>
\end{lstlisting}

\textbf{\footnotesize{Expansion definition}}

This section defines the polynomial expansions used on each composites.

\begin{lstlisting}[style=XMLStyle]
<EXPANSIONS>
    <E COMPOSITE="C[0]" NUMMODES="7" FIELDS="u" TYPE="MODIFIED" />
    <E COMPOSITE="C[1]" NUMMODES="7" FIELDS="u" TYPE="MODIFIED" />
</EXPANSIONS>
\end{lstlisting}

\textbf{\footnotesize{Conditions definition}}

This sections defines the problem solved. In this example formula and the Continuous Galerkin Method
is used as projection scheme to solve the Helmholtz equation.

\begin{lstlisting}[style=XMLStyle]
<CONDITIONS>

    <PARAMETERS>
        <P> Lambda = 1 </P>
    </PARAMETERS>

    <SOLVERINFO>
        <I PROPERTY="EQTYPE" VALUE="Helmholtz" />
        <I PROPERTY="Projection" VALUE="Continuous" />
    </SOLVERINFO>
\end{lstlisting}

Then, Dirichlet, Neumann and Robin boundary conditions are defined thanks to 
the boundary references given by the BOUNDARYREGIONS section.

\begin{lstlisting}[style=XMLStyle]
<BOUNDARYREGIONS>
        <B ID="0"> C[2] </B>
        .
        .
        <B ID="8"> C[10] </B>
    </BOUNDARYREGIONS>

    <BOUNDARYCONDITIONS>
        <REGION REF="0">
            <D VAR="u" VALUE="sin(PI*x)*sin(PI*y)" />
        </REGION>
        <REGION REF="1">
            <R VAR="u" VALUE="sin(PI*x)*sin(PI*y)-PI*sin(PI*x)*cos(PI*y)" PRIMCOEFF="1" />
        </REGION>
        <REGION REF="2">
            <N VAR="u" VALUE="(5/sqrt(61))*PI*cos(PI*x)*sin(PI*y)-(6/sqrt(61))*PI*sin(PI*x)*cos(PI*y)" />
        </REGION>
        .
        .
\end{lstlisting}

We know that for $f = -(\lambda + 2 \pi^2)sin(\pi x)cos(\pi y)$, the exact solution of the two-dimensional 
Helmholtz equation is $u = sin(\pi x)cos(\pi y)$. These functions are defined at the end of the CONDITIONS section.

\begin{lstlisting}[style=XMLStyle]
<FUNCTION NAME="Forcing">
        <E VAR="u" VALUE="-(Lambda + 2*PI*PI)*sin(PI*x)*sin(PI*y)" />
    </FUNCTION>

    <FUNCTION NAME="ExactSolution">
        <E VAR="u" VALUE="sin(PI*x)*sin(PI*y)" />
    </FUNCTION>

</CONDITIONS>
\end{lstlisting}

The exact solution will be used to evaluate the $L_2$ and $L_{inf}$ errors.

\textbf{Running the code}

The ADRSolver is used to solve the two-dimensional Helmholtz problem. To run the code, the user must first go to the directory: \\

\nekpp/builds/solvers/ADRSolver \\ 

and then copy the input file detailed above in this directory via: \\

cp ../../../regressionTests/Solvers/ADRSolver/InputFiles/Test\_Helmholtz2D\_modal.xml \\

Finally, the following command executes the solver: \\

./ADRSolver Test\_Helmholtz2D\_modal.xml \\

This execution should print out a summary of input file, the $L_2$ and $L_{inf}$ errors and the time spent on the calculation.\\

\textbf{Post-processing}

Simulation results are written in the file Test\_Helmholtz2D\_modal.fld. \nekpp provides

 post-processing tools to be able to display the results. For example, the command:\\
 
\footnotesize../../../utilities/PostProcessing/FldToGmsh Test\_Helmholtz2D\_modal.xml Test\_Helmholtz2D\_modal.fld \\
 
\normalsize 
provides the file Test\_Helmholtz2D\_modal\_u.pos in the Gmsh format which gives the following image.
 
 \begin{figure}[h!]
\begin{center}
\includegraphics[width=6cm]{Figures/Helmholtz2D}
\caption{Solution of the 2D Helmholtz Problem.}
\end{center}
\end{figure}

By writing FldToTecplot or FldToVtk instead of FldToGmsh in the previous command, Tecplot or Paraview can be used to visualize the results.



\documentclass[a4paper,10pt]{article}
\usepackage[utf8]{inputenc}

%opening
\title{User Guide - ITHACA-SEM}
\author{}

\begin{document}

\maketitle

\begin{abstract}
This guide explains functionality added with ITHACA-SEM to Nektar++. These are 
parametric model reduction routines and a few examples showing the usage.
Some familiarity with the underlying PDE-solver Nektar++ is assumed. 
\end{abstract}

\section{Installation and Testing}

-- a copy the install HowTo here

\section{Using ITHACA-SEM}

\subsection{Scope of Added Functionality}


\subsection{Setting up a Reduced Order Model}

\subsubsection{A minimal example}

\subsubsection{Overview table of parameters}

\begin{table}
\centering
\begin{tabular}{| l | l | l | l|}  
\hline
\hline
parameter & type & range  & default value \\
\hline
 \verb|load_snapshot_data_from_files|  & bool  & 0/1 & none \\%&  \\
 \verb|number_of_snapshots|  & int  & 0-$\infty$ & none \\%&  \\
 \verb|parameter_space_dimension|  & int  & 1-2 & ?? \\%&  \\
 \verb|load_cO_snapshot_data_from_files| & bool & 0/1 & 0 \\
 \verb|do_trafo_check| & bool & 0/1 & 1 \\
 \verb|compute_smaller_model_errs| &  bool & 0/1 & ?? \\
 \verb|qoi_dof|  & int  & 0-$\infty$ & -1 \\%&  \\
 \verb|POD_tolerance|  & nn  & nn & nn \\%&  \\
 \verb|ref_param_index| & nn  & nn & nn \\%&  \\
 \verb|ref_param_nu| & nn  & nn & nn \\%&  \\
 \verb|globally_connected| & nn  & nn & nn \\%&  \\
 \verb|use_Newton| & nn  & nn & nn \\%&  \\
 \verb|use_non_unique_up_to_two| & nn  & nn & nn \\%&  \\
 \verb|debug_mode| & bool  & 0/1 & 0 \\%&  \\
 \verb|write_ROM_field| & nn  & nn & nn \\%&  \\
 \verb|snapshot_computation_plot_rel_errors| & nn  & nn & nn \\%&  \\
 \verb|number_of_snapshots_dir0| & nn  & nn & nn \\%&  \\
 \verb|number_of_snapshots_dir1| & nn  & nn & nn \\%&  \\
 \verb|use_fine_grid_VV| & nn  & nn & nn \\%&  \\
 \verb|use_fine_grid_VV_and_load_ref| & nn  & nn & nn \\%&  \\
 \verb|fine_grid_dir0| & nn  & nn & nn \\%&  \\
 \verb|fine_grid_dir1| & nn  & nn & nn \\%&  \\
\hline
\hline
\end{tabular}
\caption{Overview of parameters. A default value of 'none' means that this parameter has to be set by the user.}
\label{table:parameter_overview}
\end{table} 

parameters, range, standardwert

\verb|load_snapshot_data_from_files|


\verb|number_of_snapshots|

\verb|parameter_space_dimension|

\verb|load_cO_snapshot_data_from_files|

\verb|do_trafo_check|

\verb|compute_smaller_model_errs|

\verb|qoi_dof|

\verb|POD_tolerance|

\verb|ref_param_index|

\verb|ref_param_nu|

\verb|globally_connected|

\verb|use_Newton|

\verb|use_non_unique_up_to_two|

\verb|debug_mode|

\verb|write_ROM_field|

\verb|snapshot_computation_plot_rel_errors|


\verb|number_of_snapshots_dir0|


\verb|number_of_snapshots_dir1|

\verb|use_fine_grid_VV|

\verb|use_fine_grid_VV_and_load_ref|


\verb|fine_grid_dir0|


\verb|fine_grid_dir1|





\subsubsection{Explanation of parameters}



\section{Examples}


\end{document}

\chapter{Linear elasticity solver}

\section{Synopsis}
The LinearElasticSolver is a solver for solving the linear elasticity equations
in two and three dimensions. Whilst this may be suitable for simple solid
mechanics problems, its main purpose is for use for mesh deformation and
high-order mesh generation, whereby the finite element mesh is treated as a
solid body, and the deformation is applied at the boundary in order to curve the
interior of the mesh.

Currently the following equation systems are supported:
%
\begin{center}
\begin{tabular}{lp{8cm}}
  \toprule
  Value & Description \\
  \midrule
  \inltt{LinearElasticSystem} & Solves the linear elastic equations. \\
  \inltt{IterativeElasticSystem} & A multi-step variant of the elasticity solver,
                                   which breaks a given deformation into multiple
                                   steps, and applies the deformation to a mesh. \\
  \bottomrule
\end{tabular}
\end{center}

\subsection{The linear elasticity equations}

The linear elasticity equations model how a solid body deforms under the
application of a `small' deformation or strain. The formulation starts with the
equilibrium of forces represented by the equation
%
\begin{equation}
\nabla \cdot \mathbf{S} + \mathbf{f} = \mathbf{0} \quad \textrm{in} \quad \Omega
\label{eq:strong}
\end{equation}
%
where $\mathbf{S}$ is the stress tensor and $\mathbf{f}$ denotes a
spatially-varying force. We further assume that the stress tensor $\mathbf{S}$
incorporates elastic and, optionally, thermal stresses that can be switched on
to assist in mesh deformation applications. We assume these can be decomposed so
that $\mathbf{S}$ is written as
%
\[
\mathbf{S} = \mathbf{S}_e + \mathbf{S}_t,
\]
%
where the subscripts $e$ and $t$ denote the elastic and thermal terms
respectively. We adopt the usual linear form of the elastic stress tensor as
%
\[
\mathbf{S}_e = \lambda\mbox{Tr}(\mathbf{E}) \, \mathbf{I} +\mu \mathbf{E},
\]
%
where $\lambda$ and $\mu$ are the Lam\'e constants, $\mathbf{E}$ represents the
strain tensor, and $\mathbf{I}$ is the identity tensor. For small deformations,
the strain tensor $\mathbf{E}$ is given as
%
\begin{equation}
\mathbf{E} =\frac{1}{2} \left ( \nabla \mathbf{u}+ \nabla \mathbf{u}^t \right )
\end{equation}
%
where $\mathbf{u}$ is the two- or three-dimensional vector of displacements. The
boundary conditions required to close the problem consist of prescribed
displacements at the boundary $\partial \Omega$, i.e.
\begin{equation}
  \mathbf{u} = \hat{\mathbf{u}} \quad \textrm{in}\ \partial \Omega.
\end{equation}

We further express the Lam\'e constants in terms of the Young's modulus $E$ and
Poisson ratio $\nu$ as
%
\[
\lambda = \frac{\nu E}{(1+\nu)(1-2\nu)}, \qquad \mu = \frac{E}{2(1+\nu)}.
\]
%
The Poisson ratio, valid in the range $\nu < \tfrac{1}{2}$, is a measure of the
compressibility of the body, and the Young's modulus $E > 0$ is a measure of its
stiffness.

\section{Usage}
\begin{lstlisting}[style=BashInputStyle]
LinearElasticSolver [arguments] session.xml [another.xml] ...
\end{lstlisting}



%%% Local Variables:
%%% mode: latex
%%% TeX-master: "../user-guide"
%%% End:

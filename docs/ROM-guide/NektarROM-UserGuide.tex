\documentclass[a4paper,10pt]{article}
\usepackage[utf8]{inputenc}
\usepackage{verbatim}

%opening
\title{User Guide - NektarROM}
\author{}

\begin{document}

\maketitle

\begin{abstract}
This guide explains functionality added with NektarROM to Nektar++. These are 
parametric model reduction routines and a few examples showing the usage.
Some familiarity with the underlying PDE-solver Nektar++ is assumed. 
\end{abstract}

\section{Installation and Testing}

It is integrated in Nektar++ and the 'make test' routine also covers the main
reduced order model (ROM) functionality.

\section{Scope of Added Functionality}

The ROM is activated by using in the session file

\begin{verbatim}
<I PROPERTY="EQTYPE" VALUE="SteadyOseen" />
<I PROPERTY="SolverType" VALUE="CoupledLinearisedNS_ROM" />
\end{verbatim}

\section{Using NektarROM}

The session file supports additional parameters for the IncNavierStokes module.
These are

\subsection{parameter\_space\_dimension}

\noindent type: int

\noindent default: 1

\noindent explanation: dimension of parameter space

\subsection{number\_of\_snapshots}

\noindent type: int

\noindent default: has to be set

\noindent explanation: number of snapshots for the offline sampling


\subsection{load\_snapshot\_data\_from\_files}

\noindent type: bool

\noindent default: has to be set

\noindent explanation: if set to 0, then compute the snapshots, otherwise load from file, see
\begin{verbatim}
 solvers/IncNavierStokesSolver/Tests_ROM/Test_load_files
\end{verbatim}
for usage


\subsection{param0}

\noindent type: double

\noindent default: has to be set

\noindent explanation: the first parameter value in the parameter vector in case of only one parameter direction. further paramters are numbered param1, param2, ... and so on

\subsection{POD\_tolerance}

\noindent type: double

\noindent default: has to be set

\noindent explanation: the percentage of singular values to take into account in the SVD/PCA/POD


\subsection{debug\_mode}

\noindent type: bool

\noindent default: 1

\noindent explanation: if '0' less verbose output

\subsection{type\_para1}

\noindent type: int

\noindent default: has to be set

\noindent explanation: if '0' the parameter is kinematic viscosity



   

\section{Examples}


\end{document}

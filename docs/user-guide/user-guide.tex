%%% DOCUMENTCLASS 
%%%-------------------------------------------------------------------------------

\documentclass[
a4paper, % Stock and paper size.
11pt, % Type size.
% article,
% oneside, 
onecolumn, % Only one column of text on a page.
% openright, % Each chapter will start on a recto page.
% openleft, % Each chapter will start on a verso page.
openany, % A chapter may start on either a recto or verso page.
]{memoir}

%%% PACKAGES 
%%%------------------------------------------------------------------------------

\usepackage[utf8]{inputenc} % If utf8 encoding
% \usepackage[lantin1]{inputenc} % If not utf8 encoding, then this is probably the way to go
\usepackage[T1]{fontenc}    %
\usepackage{lmodern}
\usepackage[english]{babel} % English please
\usepackage[final]{microtype} % Less badboxes

% \usepackage{kpfonts} %Font

\usepackage{amsmath,amssymb,mathtools} % Math

% \usepackage{tikz} % Figures
\usepackage{graphicx} % Include figures


%%% PAGE LAYOUT 
%%%------------------------------------------------------------------------------

\setlrmarginsandblock{0.15\paperwidth}{*}{1} % Left and right margin
\setulmarginsandblock{0.2\paperwidth}{*}{1}  % Upper and lower margin
\checkandfixthelayout

\newlength\forceindent
\setlength{\forceindent}{\parindent}
\setlength{\parindent}{0cm}
\renewcommand{\indent}{\hspace*{\forceindent}}
\setlength{\parskip}{1em}

%%% SECTIONAL DIVISIONS
%%%------------------------------------------------------------------------------

\maxsecnumdepth{subsection} % Subsections (and higher) are numbered
\setsecnumdepth{subsection}

\makeatletter %
\makechapterstyle{standard}{
  \setlength{\beforechapskip}{0\baselineskip}
  \setlength{\midchapskip}{1\baselineskip}
  \setlength{\afterchapskip}{8\baselineskip}
  \renewcommand{\chapterheadstart}{\vspace*{\beforechapskip}}
  \renewcommand{\chapnamefont}{\centering\normalfont\Large}
  \renewcommand{\printchaptername}{\chapnamefont \@chapapp}
  \renewcommand{\chapternamenum}{\space}
  \renewcommand{\chapnumfont}{\normalfont\Large}
  \renewcommand{\printchapternum}{\chapnumfont \thechapter}
  \renewcommand{\afterchapternum}{\par\nobreak\vskip \midchapskip}
  \renewcommand{\printchapternonum}{\vspace*{\midchapskip}\vspace*{5mm}}
  \renewcommand{\chaptitlefont}{\centering\bfseries\LARGE}
%  \renewcommand{\printchaptertitle}[1]{\chaptitlefont ##1}
  \renewcommand{\afterchaptertitle}{\par\nobreak\vskip \afterchapskip}
}
\makeatother

%\chapterstyle{standard}
\chapterstyle{madsen}

\setsecheadstyle{\normalfont\large\bfseries}
\setsubsecheadstyle{\normalfont\normalsize\bfseries}
\setparaheadstyle{\normalfont\normalsize\bfseries}
\setparaindent{0pt}\setafterparaskip{0pt}

%%% FLOATS AND CAPTIONS
%%%------------------------------------------------------------------------------

\makeatletter                  % You do not need to write [htpb] all the time
\renewcommand\fps@figure{htbp} %
\renewcommand\fps@table{htbp}  %
\makeatother                   %

\captiondelim{\space } % A space between caption name and text
\captionnamefont{\small\bfseries} % Font of the caption name
\captiontitlefont{\small\normalfont} % Font of the caption text

\changecaptionwidth          % Change the width of the caption
\captionwidth{1\textwidth} %

%%% ABSTRACT
%%%------------------------------------------------------------------------------

\renewcommand{\abstractnamefont}{\normalfont\small\bfseries} % Font of abstract title
\setlength{\absleftindent}{0.1\textwidth} % Width of abstract
\setlength{\absrightindent}{\absleftindent}

%%% HEADER AND FOOTER 
%%%------------------------------------------------------------------------------

\makepagestyle{standard} % Make standard pagestyle

\makeatletter                 % Define standard pagestyle
\makeevenfoot{standard}{}{}{} %
\makeoddfoot{standard}{}{}{}  %
\makeevenhead{standard}{\bfseries\thepage\normalfont\qquad\small\leftmark}{}{}
\makeoddhead{standard}{}{}{\small\rightmark\qquad\bfseries\thepage}
% \makeheadrule{standard}{\textwidth}{\normalrulethickness}
\makeatother                  %

\makeatletter
\makepsmarks{standard}{
\createmark{chapter}{both}{shownumber}{\@chapapp\ }{ \quad }
\createmark{section}{right}{shownumber}{}{ \quad }
\createplainmark{toc}{both}{\contentsname}
\createplainmark{lof}{both}{\listfigurename}
\createplainmark{lot}{both}{\listtablename}
\createplainmark{bib}{both}{\bibname}
\createplainmark{index}{both}{\indexname}
\createplainmark{glossary}{both}{\glossaryname}
}
\makeatother                               %

\makepagestyle{chap} % Make new chapter pagestyle

\makeatletter
\makeevenfoot{chap}{}{\small\bfseries\thepage}{} % Define new chapter pagestyle
\makeoddfoot{chap}{}{\small\bfseries\thepage}{}  %
\makeevenhead{chap}{}{}{}   %
\makeoddhead{chap}{}{}{}    %
% \makeheadrule{chap}{\textwidth}{\normalrulethickness}
\makeatother

\nouppercaseheads
\pagestyle{standard}               % Choosing pagestyle and chapter pagestyle
\aliaspagestyle{chapter}{chap} %

%%% NEW COMMANDS
%%%-----------------------------------------------------------------------------

\newcommand{\p}{\partial} %Partial
% Or what ever you want


%%% CODE SNIPPETS, COMMANDS, ETC
%%%-----------------------------------------------------------------------------
\usepackage{xcolor}
\usepackage{listings} % Display code / shell commands
%\newcommand{\shellcommand}[1]{\begin{lstlisting} \#1 \end{lstlisting}
\lstdefinestyle{BashInputStyle}{
  language=bash,
  basicstyle=\small\sffamily,
%  numbers=left,
%  numberstyle=\tiny,
%  numbersep=3pt,
  frame=,
  columns=fullflexible,
  backgroundcolor=\color{black!10},
  linewidth=0.9\linewidth,
  xleftmargin=0.1\linewidth
}

%%% TABLE OF CONTENTS
%%%-----------------------------------------------------------------------------

\maxtocdepth{subsection} % Only parts, chapters and sections in the table of contents
\settocdepth{subsection}

%\AtEndDocument{\addtocontents{toc}{\par}} % Add a \par to the end of the TOC

%%% INTERNAL HYPERLINKS
%%%-------------------------------------------------------------------------------

\usepackage{hyperref}   % Internal hyperlinks
\hypersetup{
pdfborder={0 0 0},      % No borders around internal hyperlinks
pdfauthor={I am the Author} % author
}
\usepackage{memhfixc}   %

%%% THE DOCUMENT
%%% Where all the important stuff is included!
%%%-------------------------------------------------------------------------------

\author{Department of Aeronautics, Imperial College London, UK\\
Scientific Computing and Imaging Institute, University of Utah, USA}
\title{Nektar++: Spectral/hp Element Framework}

\begin{document}

\frontmatter

\maketitle

\clearpage

\tableofcontents*
\clearpage

\chapter{Introduction}


\mainmatter

\chapter{Installation}

\section{Installing on Debian/Ubuntu Systems}

\section{Installing of Redhat/Fedora Systems}

\section{Installing from Source}


\chapter{Solvers}


\chapter{XML Input File Reference}


\chapter{Command-line Options}

%\begin{lstlisting}
%--verbose
%\end{lstlisting}
\lstinline[style=BashInputStyle]{--verbose}\\
\hangindent=1.5cm
Displays extra info.

\lstinline[style=BashInputStyle]{--version}\\
\hangindent=1.5cm
Displays software version, and source control information if applicable.

\lstinline[style=BashInputStyle]{--help}\\
\hangindent=1.5cm
Displays help information about the available command-line options for the executable.

\lstinline[style=BashInputStyle]{--parameter [key]=[value]}\\
\hangindent=1.5cm
Override a parameter (or define a new one) specified in the XML file.

\lstinline[style=BashInputStyle]{--solverinfo [key]=[value]}\\
\hangindent=1.5cm
Override a solverinfo (or define a new one) specified in the XML file.

\lstinline[style=BashInputStyle]{--shared-filesystem}\\
\hangindent=1.5cm
By default when running in parallel the complete mesh is loaded by all processes, although partitioning is done uniquely on the root process only and communicated to the other processes. Each process then writes out its own partition to the local working directory. This is the most robust approach in accounting for systems where the distributed nodes do not share a common filesystem. In the case that there is a common filesystem, this option forces only the root process to load the complete mesh, perform partitioning and write out the session files for all partitions. This avoids potential memory issues when multiple processes attempt to load the complete mesh on a single node.

\lstinline[style=BashInputStyle]{--npx [int]}\\
\hangindent=1.5cm
When using a fully-Fourier expansion, specifies the number of processes to use in the x-coordinate direction.

\lstinline[style=BashInputStyle]{--npy [int]}\\
\hangindent=1.5cm
\quad When using a fully-Fourier expansion or 3D expansion with two Fourier directions, specifies the number of processes to use in the y-coordinate direction.

\lstinline[style=BashInputStyle]{--npz [int]}\\
\hangindent=1.5cm
When using Fourier expansions, specifies the number of processes to use in the z-coordinate direction.

\lstinline[style=BashInputStyle]{--part-info}\\
\hangindent=1.5cm
Prints detailed information about the generated partitioning, such as number of
elements, number of local degrees of freedom and the number of boundary degrees
of freedom.

\lstinline[style=BashInputStyle]{--part-only [int]}\\
\hangindent=1.5cm
Partition the mesh only into the specified number of partitions, write to file
and exit. This can be used to pre-partition a very large mesh on a single
high-memory node, prior to being executed on a multi-node cluster.

\lstinline[style=BashInputStyle]{--use-metis}\\
\hangindent=1.5cm
Forces the use of METIS for mesh partitioning. If \nekpp{} is compiled with
Scotch support, the default is to use Scotch.

\lstinline[style=BashInputStyle]{--use-scotch}\\
\hangindent=1.5cm
Forces the use of Scotch for mesh partitioning.


\chapter{Frequently Asked Questions}


%\appendix


\backmatter

%%% BIBLIOGRAPHY
%%% -------------------------------------------------------------

% \bibliographystyle{utphysics}
% \bibliography{ref}

\end{document}
